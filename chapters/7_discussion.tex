\section{Discussion}
To develop and determine an accurate positioning system is a challenge on many levels as it requires computationally advanced electronic components that can perform highly accurate calculations. The difficulties faced when solving this problem arise when the determining the orientation of an object as it produces many sources of error. The downside of having a well oriented system is at the expense of detection of low velocities and vise versa. The positioning system performs at high accuracy (cm precision) during non-rotational movements but has trouble with drift when the object is rotated. This is due to the low convergence rate of the orientation filter which is clearly a big system limitation. A suggestion on solving this problem could be making the tuning parameters of the filter more dynamic which would make the system more adaptable to extreme rotations and low velocities.

Common problems faced when trying to implement an accurate position lies in the gyroscope because it tends to drift significantly over time. The observations made during the studying of other positioning systems have been that they all suffer in accuracy due the gyroscope drift. These findings lead to the consideration of reducing the gyroscope drift significantly. Methods and algorithms have been presented in this thesis which evidently reduced the drift components and these methods have helped increase the accuracy of the positioning system. Similarly to other systems, this system suffers from inaccuracies. Due to the gyroscope drift reduction algorithms developed, this system stands out from other previously developed systems in that sense. To develop a very accurate positioning system using low cost MEMS sensors is a challenging subject that comes with many complications. The system requirements are very strict due to the environmental, economical and accuracy specifications. Therefore implementing a system of very high accuracy, low cost and high mobility is very difficult. In order to achieve a higher accuracy, the electronic components need to have a higher performance, higher resolution and a higher sampling frequency. The micro controller unit which is the heart of the whole system needs to have a faster processor but in the end, it is the sensors that limit the system’s performance.