\chapter*{Resumo}
\justify
Com o recente surgimento de sistemas microeletromecânicos amplamente acessíveis e precisos nos últimos anos, o rastreio de movimento através de sistemas de inerciais tornou-se omnipresente nos dias de hoje. Contrariamente à localização baseada no Sistema de Posicionamento Global (GPS), os Sistemas de Navegação Inercial (SNI) não são afetados intrinsecamente pela interferência de sinal, suscetibilidades de bloqueio e falsificação. As medições dos sensores inerciais também são adquiridas a elevadas taxas de amostragem e podem ser integradas numericamente para estimar os conhecimentos de posição e orientação. Estas medições são precisas numa escala de pequena dimensão, mas acumulam gradualmente erros durante longos períodos. Combinar múltiplos sensores inerciais num método conhecido como fusão de sensores permite produzir uma mais consistente e confiável compreensão do sistema, diminuindo erros acumulativos. Vários algoritmos de fusão de sensores ocorrem na literatura com o objetivo de estimar os Sistemas de Referência de Atitude e Rumo (SRAR) de um corpo rígido no que diz respeito a uma estrutura de referência. Este trabalho descreve o desenvolvimento e implementação de um sistema multiusos de baixo custo para estimativa de posição e orientação. Além disso, apresenta uma comparação experimental de uma série de soluções de fusão de sensores e compara o seu desempenho na estimativa da posição de um objeto em movimento.

\keywords{ Fusão de Sensores  \and Sistemas de navegação Inercial \and Sistemas Microeletromecânicos \and Unidade de Medição Inercial \and Sistemas de Referência de Atutude e Rumo }
